\section{Recursão}

\begin{frame}
    \frametitle{Recursão}

    \begin{itemize}
        \item A definição de um objeto é dita {recursiva} se ela se 
        refere ao {próprio objeto} a ser definido
        
        \item Uma definição recursiva tem {duas} partes: o
        {caso base} e as {regras de construção} 

        \item No {caso base}, todos os {elementos estruturais 
        básicos} são listados

        \item As {regras de construção} determinam como os elementos
        estruturais se relacionam, e envolvem o {próprio objeto}

        \item Uma função é dita recursiva se o cálculo do seu retorno envolve a
            chamada da própria função
    \end{itemize}

\end{frame}

\begin{frame}

    \frametitle{Exemplo de definição recursiva: números naturais}

    \begin{itemize}
        \item {caso base:} $0\in N$ 

        \item {regra de construção:} se $n\in N$, então $(n+1)\in N$

        \item {definição de escopo:} {não há} outros elementos no 
        conjunto $N$ (isto é, não há {outras formas} de se obter números
        naturais)
    \end{itemize}

\end{frame}

\begin{frame}
    \frametitle{Exemplo de função recursiva: fatorial de um número natural}

    \begin{itemize}
        \item O fatorial do número natural $n$ é dado por
    $$
    n! = \left\lbrace \begin{array}{ll} 1, & \mbox{se}\, n = 0 \\ n\cdot (n - 1)!, & \mbox{se}\, n > 0 \end{array}\right.
    $$ 
        \item Observe que a primeira parte da definição corresponde ao caso base

        \item Já a segunda parte se refere à regra de construção

        \item A definição ade escopo fica implícita: $n$ deve ser um número natural

        \item A implementação em C/C++ é quase idêntica à definição:

            \inputcode{cpp}{factorial.cpp}

    \end{itemize}
\end{frame}

\begin{frame}
    \frametitle{Relações de pertinência}

    \begin{itemize}
        \item É possível determinar se um elemento {pertence ou não} 
        a um conjunto de elementos definidos recursivamente aplicando a 
        regra de construção {sucessivamente}, {decompondo} o
        elemento em outros elementos do conjunto

        \item Por exemplo, para $n = 5$,
        \[
        \begin{array}{l}
            5 = 4 + 1, \, \, \mbox{logo se}\, \, 4\in N, \, \, \mbox{então}\, \,
            5 \in N \\  
            4 = 3 + 1, \, \, \mbox{logo se}\, \, 3\in N, \, \, \mbox{então}\, \,
            4 \in N \\  
            3 = 2 + 1, \, \, \mbox{logo se}\, \, 2\in N, \, \, \mbox{então}\, \,
            3 \in N \\  
            2 = 1 + 1, \, \, \mbox{logo se}\, \, 1\in N, \, \, \mbox{então}\, \,
            2 \in N \\  
            1 = 1 + 0 
        \end{array}
        \]
        \item Como $0\in N$, então $1, 2, 3, 4,5 \in N$
    \end{itemize}
\end{frame}

\begin{frame}
    \frametitle{Resolução de recorrências}

    \begin{itemize}
        \item Para encontrar o $n$-ésimo termo do conjunto, é necessário encontrar
        todos os termos anteriores que se fizerem presentes de acordo com as regras
        de construção

        \item Este processo impacta diretamente na complexidade computacional da rotina

        \item Onde for possível, a troca de uma definição recursiva por uma 
        definição que seja independente dependa de termos anteriores
        leva a implementações {mais eficientes}
        
        \item Por exemplo, a definição {recursiva}
        \[
            g(n) = \left\lbrace \begin{array}{ll} 1, & \mbox{se}\, n = 0 \\ 
            2g(n - 1), & \mbox{se}\, n > 0 \end{array}\right.
        \] 
         equivale a definição 
        \[ g(n) = 2^n \]
    \end{itemize}

\end{frame}

\begin{frame}[fragile]{Resolução de recorrência por hipótese e demonstração}

    \begin{itemize}
        \item Uma maneira de se resolver uma relação de recorrência é observar alguns termos,
            formular uma hipótese e demonstrar esta hipótese usando indução matemática

        \item Por exemplo, se $h(n) = 2h(n - 1) + 1$, com $h(1) = 1$, então
        \[
            h(2) = 3,\ h(3) = 7,\ h(4) = 15,\ \ldots
        \]

        \item Pode se observar que, para estes termos, é possível formular a hipótese de que
            $h(n) = 2^n - 1$

        \item Para provar esta hipótese, primeiramente deve-se verificar o caso base:
        \[
            h(1) = 2^1 - 1 = 1,
        \]
        de modo que a hipótese vale para o caso base

    \end{itemize}

\end{frame}

\begin{frame}[fragile]{Resolução de recorrência por hipótese e demonstração}

    \begin{itemize}
        \item Suponha que a hipótese seja verdadeira para $m\in N$. Daí, usando a definição
            recursiva de $h(n)$ e a hipótese, segue que
        \[
            h(m + 1) = 2h(m) + 1 = 2(2^m - 1) + 1 = 2^{m + 1} - 2 + 1 = 2^{m + 1} - 1,
        \]
        de modo que $h(m + 1)$ é verdadeira sempre que $h(m)$ for verdadeira

        \item Assim, de acordo com o Princípio da Indução, a hipótese é verdadeira

        \item Observe que, se a hipótese fosse falsa, ou ela falharia no caso base, ou não
            seria possível provar a segunda parte da indução, isto é, que a veracidade para
            $m$ implica a veracidade para $m + 1$
    \end{itemize}

\end{frame}

\begin{frame}[fragile]{Resolução de recorrência por expansão}

    \begin{itemize}
        \item Outra maneira de resolver uma relação de recorrência é aplicar a
            recorrência em si mesmo, expandindo sua definição

        \item Utilizando a mesma definição de $h(n)$, tem-se que
        \begin{align*}
            h(n) &= 2h(n - 1) + 1 \\
            &= 2(2h(n - 2) + 1) + 1 = 4h(n - 2) + 3\\
            &= 4(2h(n - 3) + 1) + 3 = 8h(n - 3) + 7\\
            &\ldots
        \end{align*}

        \item Esta expansão leva a nova hipótese de que
        \[
            h(n) = 2^kh(n - k) + (2^k - 1),
        \]
        para $k$ natural

        \item Esta hipótese pode ser provada por indução

        \item Fazendo $k = (n - 1)$ e lembrando que $h(1) = 1$, segue que $h(n) = 2^n - 1$
    \end{itemize}

\end{frame}
