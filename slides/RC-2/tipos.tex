\section{Tipos de dados compostos e de usuário}

\subsection{Estruturas}

\begin{frame}[fragile]

	\frametitle{Estruturas}

    \metroset{block=fill}
    \begin{block}{Sintaxe para declaração de estruturas}
        \inputsyntax{c}{struct.st}
    \end{block}

	\begin{itemize}
		\item Uma estrutura em C/C++ consiste numa coleção de variáveis 
		referenciadas por um nome comum

		\item As variáveis que compõem a estrutura são denominadas membros, campos ou elementos
		da estrutura.

		\item É possível definir variáveis do tipo da estrutura criada

		\item Os membros de uma instância de um estrutura são acessados da seguinte forma: 
			\code{c}{nome_da_instancia.membro}

		\item O tamanho de uma estrutura é dado pela soma do tamanho de todos os seus membros, 
        mas pode ser arredondado para cima para o múltiplo mais próximo do tamanho de uma 
        palavra do processador
	\end{itemize}

\end{frame}

\begin{frame}[fragile]{Exemplo de uso de estrutura}
    \inputcode{c}{cadastro.c}
\end{frame}

\begin{frame}[fragile]{Uniões}

    \metroset{block=fill}
    \begin{block}{Sintaxe para declaração de uniões}
        \inputsyntax{c}{union.st}
    \end{block}
	\begin{itemize}
		\item Uma união consiste numa área de memória compartilhada por duas ou mais variáveis

		\item As variáveis que compõem a união são denominadas membros, campos ou elementos 
        da união

		\item É possível definir variáveis do tipo da união criada

		\item Os membros da união são acessados com a mesma sintaxe utilizada nas estruturas

		\item O tamanho de uma união é igual ao maior 
		dentre todos os tamanhos dos campos da união, e pode
		ser arredondado de forma semelhante às estruturas
	\end{itemize}

\end{frame}

\begin{frame}[fragile]{Exemplo de uso de união}
    \inputcode{c}{rotate.c}
\end{frame}

\begin{frame}[fragile]{Classes}

    \metroset{block=fill}
    \begin{block}{Sintaxe para declaração de classes}
        \inputsyntax{c++}{class.st}
    \end{block}
	
    Uma classe consiste numa agregação de dados que tem uma relação comum, e num conjunto de 
    funções que agem sobre estes dados.

\end{frame}

\begin{frame}[fragile]{Classes}

	\begin{itemize}
		\item As variáveis que compõem a classe são denominadas membros da classe, enquanto as 
        função são denominadas métodos da classe

		\item É possível definir variáveis do tipo da classe criada. Estas variáveis são 
        denominadas objetos ou instâncias da classe

		\item Os membros de uma instância de uma classe ou seus métodos são acessados usando a
        sintaxe: \code{c++}{instancia.{membro|metodo}}

		\item Apenas as variáveis e os métodos públicos podem ser acessados fora do escopo da 
        classe

        \item Funções amigas podem acessar as seções privadas e protegidas da classe
	\end{itemize}

\end{frame}

\begin{frame}[fragile]{Exemplo de uso de classe}
    \inputcode{cpp}{complex.h}
\end{frame}

\begin{frame}[fragile]{Exemplo de uso de classe}
    \inputsnippet{cpp}{1}{21}{complex.cpp}
\end{frame}

\begin{frame}[fragile]{Exemplo de uso de classe}
    \inputsnippet{cpp}{23}{42}{complex.cpp}
\end{frame}

\begin{frame}[fragile]{Exemplo de uso de classe}
    \inputcode{cpp}{testcomplex.cpp}
\end{frame}
