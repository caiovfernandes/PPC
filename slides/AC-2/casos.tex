\section{Pior caso, melhor caso, caso médio}

\begin{frame}[fragile]{Definição}

	\begin{itemize}
        \item A análise de algoritmos considera três cenários possíveis, os quais relacionam
        a entrada com o número de iterações do algoritmo

		\item O pior caso acontece quando o algoritmo é executado o número máximo de iterações

		\item No melhor caso o algoritmo é executado o número mínimo de iterações possível

		\item O caso médio representa o cenário esperado quando as entradas possuem determinada
        distribuição de probabilidade de ocorrência
		caso

		\item Em termos formais, a complexidade do caso médio $C_M$ é dada por
		\[
		C_M = \sum_i p(\mbox{input}_i)\mbox{steps}(\mbox{input}_i)
		\] 
        com $p(\mbox{input}_i) \geq 0$ e $\sum_i p(\mbox{input}_i) = 1$

		\item A fórmula para o caso médio coincide com a definição probabilística de
		valor esperado

	\end{itemize}

\end{frame}

\begin{frame}[fragile]{Observações}

    \begin{itemize}
        \item O melhor caso tem interesse meramente teórico, não sendo levado em
        consideração na maior parte das análises

        \item A maioria das análises se concentram no pior caso, pois ele é uma estimava
        de como o algoritmo efetivamente vai se comportar
    
        \item Embora o caso médio seja mais próximo da realidade, sua análise é mais técnica
        e depende de conceitos elaborados de matemática e probabilidade

        \item Além disso, o caso médio tende a ser idêntico ao pior caso no contexto da
        complexidade assintótica

        \item A notação mais utilizada é a notação Big-$O$, seguida pela notação
        Big-$\Theta$
    \end{itemize}

\end{frame}
