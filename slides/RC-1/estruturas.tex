\section{Estruturas de seleção e controle}

\begin{frame}[fragile]{\texttt{if/else}}

    \metroset{block=fill}
    \begin{block}{Sintaxe para declaração do \texttt{if}}
        \inputsyntax{c}{if.st}
    \end{block}
	\begin{itemize}
	    \item Se a condição for verdadeira, o bloco \code{c}{if} será executado

        \item Caso contrário, o bloco \code{c}{else} será executado, se existir

		\item Operador ternário: a variável assume um dos dois valores listados, a depender 
        do valor lógico da condição: \texttt{A}, se verdadeira; \texttt{B}, se falsa.

        \item Sintaxe: \texttt{[variável = ] condição ? A : B;}

        \item Ambos valores \texttt{A} e \texttt{B} devem ter o mesmo tipo
	\end{itemize}

\end{frame}

\begin{frame}[fragile]{Exemplo de uso de \texttt{if/else}}
    \inputcode{cpp}{sinal.cpp}
\end{frame}

\begin{frame}[fragile]{\texttt{switch}}

    \metroset{block=fill}
    \begin{block}{Sintaxe para declaração do \texttt{switch}}
        \inputsyntax{c}{switch.st}
    \end{block}

\end{frame}

\begin{frame}[fragile]{\texttt{switch}}

	\begin{itemize}

		\item Se \texttt{valor} for igual a um dos valores 
		\texttt{valor1, ..., valorN} descritos, serão executados os comandos
		que se seguem a cláusula \code{c}{case} associada até 
		que se encontre um comando \code{c}{break} ou termine 
		o bloco do \code{c}{switch}

		\item A cláusula \code{c}{default}, que é escolhida 
		caso {valor} não corresponda a nenhum valor listado, é 
		{opcional}

		\item O \texttt{valor} indicado deve ser do tipo inteiro 
		(\code{c}{char} ou \code{c}{int})
	\end{itemize}

\end{frame} 


\begin{frame}[fragile]{Exemplo de uso de \texttt{switch}}
    \inputsnippet{c}{1}{21}{mencao.c}
\end{frame}

\begin{frame}[fragile]{Exemplo de uso de \texttt{switch}}
    \inputsnippet{c}{23}{45}{mencao.c}
\end{frame}

\begin{frame}[fragile]{\texttt{for}}
  
    \metroset{block=fill}
    \begin{block}{Sintaxe para declaração do \texttt{for}}
        \inputsyntax{c}{for.st}
    \end{block}
	\begin{itemize}
		\item O laço \code{c}{for} é começa com a etapa de 
		\texttt{inicialização}
    
        \item Em seguida, é verificada a \texttt{condição}: se falsa, o laço é encerrado; caso 
        contrário, é executado o \texttt{bloco de comandos}

		\item Finalizada a execução do \texttt{bloco de comandos}, é executado o {\tt incremento} 
        e a condição é novamente testada, e assim sucessivamente até o encerramento do laço
	\end{itemize}

\end{frame}

\begin{frame}[fragile]{\texttt{for}}

	\begin{itemize}
		\item A {\tt inicialização}, a {\tt condição} e o 
		{\tt incremento} são opcionais

		\item O laço \code{c}{for}, assim como os demais laços, 
		pode ser interrompido a qualquer momento através de um 
		comando \code{c}{break}

		\item O comando \code{c}{continue} força o encerramento 
		prematuro do bloco de comandos, levando a execução imediatamente para a avaliação
		do {\tt incremento}

        \item Sem o uso de um \code{c}{break}, o laço poderá se executado infinitamente
        (o que resultará no travamento do console ou numa falha de segmentação) caso a 
        condição não se torne verdadeira em algum ponto do laço
	\end{itemize}

\end{frame}


\begin{frame}[fragile]{Exemplo de uso do \texttt{for}}
    \inputcode{cpp}{primos.cpp}
\end{frame}

\begin{frame}[fragile]{\texttt{while}}

    \metroset{block=fill}
    \begin{block}{Sintaxe para declaração do \texttt{while}}
        \inputsyntax{c}{while.st}
    \end{block}
	\begin{itemize}
		\item O laço \code{c}{while} começa com a verificação da {\tt condição}: se 
        falsa, o laço é encerrado; caso contrário, é executado o \texttt{bloco de comandos} 

		\item Finalizada a execução do \texttt{bloco de comandos}, a 
		condição é novamente testada, e assim sucessivamente,
		até o encerramento do laço

		\item O laço \code{c}{while}, assim como os demais 
		laços, pode ser interrompido a qualquer momento através de 
		um comando \code{c}{break}

		\item O comando \code{c}{continue} força o encerramento 
		prematuro do bloco de comandos, levando a execução 
		imediatamente para a avaliação da \textbf{condição}

	\end{itemize}

\end{frame}

\begin{frame}[fragile]{Exemplo de uso de \texttt{while}}
    \inputcode{c}{fibonacci.c}
\end{frame}
