\section{Recursos para prática de programação competitiva}

\begin{frame}[fragile]{Codeforces}

    \begin{itemize}
        \item Codeforces (\texttt{codeforces.com}) é um juiz online que hospeda competições
            regularmente

        \item É, atualmente, o melhor juiz online para treinamento de alto desempenho

        \item Tem como vantagem a transparência: as soluções dos problemas de todos os
            participantes ficam em aberto para estudo, e também são publicados editoriais
            com comentários sobre as soluções dos problemas

        \item Os contests podem ser feitos posteriormente, através da opção de participação
            virtual

        \item Vale a pena fazer ao menos os contests educacionais virtualmente

    \end{itemize}

\end{frame}

\begin{frame}[fragile]{UVA Online Judge}

    \begin{itemize}
        \item O UVA (\textit{uva.onlinejudge.org}) é um juiz online com uma imensa base de
            problemas

        \item Já foi a principal plataforma de treino, mas tem sido abandonado em favor do
            Codeforces

        \item Contudo, os problemas do UVA são mais próximos dos problemas do ACM ICPC,
            de forma que um participante de alto nível deve ter contato com estes problemas
            também

        \item A plataforma uHunt (\texttt{uhunt.onlinejudge.org}) sistematiza a apresentação
            dos problemas do UVA, e lista também o progresso do usuário nos problemas listados
            no CP3

        \item O uHunt também permite a criação de contests virtuais

        \item O Live Archive (\texttt{https://icpcarchive.ecs.baylor.edu/}) traz os problemas
            da maioria dos eventos do ACM ICPC
    \end{itemize}

\end{frame}

\begin{frame}[fragile]{Livros Didáticos}

    \begin{itemize}
        \item O livro \textit{Competitive Programming 3}, dos irmãos Halim, foi a grande
            referência de programação competitiva da época

        \item Ele traz uma série de exercícios sugeridos do UVA, que estão listados no uHunt

        \item O livro \textit{Competitive Programmer's Handbook}, de Antti Laaksonen, é um
            livro mais recente (2018), e que traz estruturas e algoritmos ausentes no CP3

        \item Além disso, o PDF com o livro é gratuito

        \item Contudo, ele não dá implementação completas dos algoritmos, deixando-as a cargo
            dos leitores
    \end{itemize}

\end{frame}
