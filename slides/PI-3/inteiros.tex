\section{Inteiros}

\begin{frame}[fragile]{Variáveis integrais}

    \begin{itemize}
        \item Em C/C++ há 3 tipos de dados integrais: \code{c++}{char, short, int}

        \item Embora o tipo \code{c++}{char} seja utilizado para representar um caractere,
            efetivamente ele é capaz de armazenar, em geral, um inteiro de 8 \textit{bits}

        \item Os tamanhos característicos dos tipos \code{c++}{short} e \code{c++}{int} são
            de 16 e 32 \textit{bits}, respectivamente

        \item A extensão \code{c++}{long long}, usada para representar inteiros de 64 
            \textit{bits}, foi incorporado ao C++ a partir do padrão C++11

        \item Os modificadores \code{c++}{signed} e \code{c++}{unsigned} modificam o intervalo
            de valores representáveis pelos tipos integrais
    \end{itemize}

\end{frame}

\begin{frame}[fragile]{Faixa de valores das variáveis integrais}

    \begin{table}[!ht]
        \centering

        \begin{tabular}{lcr}
            \toprule
            \textbf{Tipo} & \textit{bits} & \textbf{Intervalo} \\
            \midrule
            \texttt{\textbf{char}} & 8 & $[-128, 127]$ \\
            \rowcolor[gray]{0.9}
            \texttt{\textbf{unsigned char}} & 8 & $[0, 255]$ \\
            \texttt{\textbf{short}} & 16 & $[-32768, 32677]$ \\
            \rowcolor[gray]{0.9}
            \texttt{\textbf{unsigned short}} & 16 & $[0, 65535]$ \\
            \texttt{\textbf{int}} & 32 & $\approx [-2\times 10^9, 2\times 10^9]$ \\
            \rowcolor[gray]{0.9}
            \texttt{\textbf{unsigned int}} & 32 & $\approx [0, 4\times 10^9]$ \\
            \texttt{\textbf{long long}} & 64 & $\approx [-9\times 10^{18}, 9\times 10^{18}]$ \\
            \rowcolor[gray]{0.9}
            \texttt{\textbf{unsigned long long}} & 64 & $\approx [0, 10^{19}]$ \\
            \bottomrule
        \end{tabular}
    \end{table}

\end{frame}

\begin{frame}[fragile]{Observações sobre as variáveis integrais}

    \begin{itemize}
        \item Nas variáveis sinalizadas, valores negativos tem o \textit{bit} mais significativo
            ligado

        \item Nas variáveis integrais existem divisores de zero, isto é, valores diferentes de
            zero cujo produto é igual a zero:
            \inputcode{c++}{zero.cpp}

        \item Isto acontece por conta do \textit{overflow}: o resultado não cabe na faixa de
            valores representáveis, de modo que o \lq\lq excesso\rq\rq\ é descartado

        \item A multiplicação de dois inteiros, em geral, resulta em \textit{overflow}: nestes
            casos, o ideal é converter ambos operando para \code{c++}{long long} e guarda o
            resultado também no tipo \code{c++}{long long}:
            \inputcode{c++}{overflow.cpp}
    \end{itemize}

\end{frame}

\begin{frame}[fragile]{Aritmética estendida}

    \begin{itemize}
        \item Para armazenar valores inteiros que excedem o limite de variáveis do tipo
            \code{c++}{long long}, é necessário o uso de aritmética estendida

        \item Até a versão C++17, o C++ não tem suporte nativo à aritemética estendida

        \item Uma alternativa é utilizar o Python 3:
            \inputcode{python3}{python.py3} 

        \item Outra alternativa é recorrer à classe \code{java}{BigInteger} do Java:
            \inputcode{java}{BigInt.java}
    \end{itemize}

\end{frame}
