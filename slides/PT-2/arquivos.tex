\section{Ponteiros e arquivos}

\begin{frame}[fragile]{Leitura e escrita em arquivos em C}

	\begin{itemize}
		\item Em C, a manipulação de arquivos é feita através da estrutura \code{c}{FILE}
         e as funções associadas a ela, que fazem parte da biblioteca \code{c}{stdio.h}

		\item A abertura de arquivos em C é feita através da função \code{c}{fopen()}, cuja 
        assinatura é
        \inputsyntax{c}{fopen.st}
		\item Os parâmetros da função \code{c}{fopen()} são o caminho para o arquivo (\code{c}{path}) 
        e o modo de abertura: \lq\lq \code{c}{r}\rq\rq\ para {leitura}, \lq\lq \code{c}{w}\rq\rq\ 
        para escrita e \lq\lq \code{c}{a}\rq\rq\ para anexar

		\item O retorno é um ponteiro para a estrutura \code{c}{FILE}, que deve ser passada         para as diversas funções que realizam a leitura e a escrita de dados em arquivo, além das 
        funções de posicionamento do cursor

	\end{itemize}

\end{frame}

\begin{frame}[fragile]{Leitura e escrita em arquivos em C}

	\begin{itemize}
		\item Entre as funções de escrita, as mais comuns são: \code{c}{fprintf(), fputc(), fwrite()}

		\item Entre as funções de leitura, as mais comuns são: \code{c}{fscanf(), fgetc(), fread()}

		\item As funções para manipulação do cursor são: \code{c}{fseek(), ftell(), rewind()}

		\item Ao finalizar o uso de um arquivo, a estrutura \code{c}{FILE} deve ser 
        desalocada através da chamada da função \code{c}{fclose()}

		\item Em C++, a manipulação de arquivos é feita pelas classes da biblioteca \code{c}{fstream}

        \item Uma vez inicializados os fluxos de entrada e saída da \code{c}{fstream}, a leitura
        e a escrita são feitas da mesma forma que são feitas com as classes \code{c}{cin} e 
        \code{c}{cout}
	\end{itemize}

\end{frame}

\begin{frame}[fragile]{Exemplo de manipulação de arquivos em C}
    \inputcode{c}{cripto.h}
\end{frame}

\begin{frame}[fragile]{Exemplo de manipulação de arquivos em C}
    \inputcode{c}{cripto.c}
\end{frame}

\begin{frame}[fragile]{Exemplo de manipulação de arquivos em C}
    \inputsnippet{c}{1}{21}{criptofile.c}
\end{frame}

\begin{frame}[fragile]{Exemplo de manipulação de arquivos em C}
    \inputsnippet{c}{23}{42}{criptofile.c}
\end{frame}

\begin{frame}[fragile]{Exemplo de manipulação de arquivos em C}
    \inputsnippet{c}{43}{63}{criptofile.c}
\end{frame}


